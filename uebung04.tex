\documentclass{uebblatt}

\begin{document}

\maketitle{4}

\begin{aufgabe}{Ein zweiter Ausblick auf Verzweigung von Primidealen}
Sei~$\qqq \defeq (11) \subseteq \O_K$ mit~$K \defeq \QQ[\sqrt{-5}]$.
\begin{multicols}{2}
\begin{enumerate}
\item Zeige, dass~$\qqq$ ein Primideal ist.
\item Berechne den Grad~$[\O_K/\qqq : \ZZ/(11)]$.
\end{enumerate}
\end{multicols}
\vspace{-1em}
\end{aufgabe}

\begin{aufgabe}{Ein Kriterium für die Trägheit eines Primideals}
Sei~$p$ eine Primzahl. Sei~$d$ quadratfrei und~$K \defeq \QQ[\sqrt{d}]$.
\begin{enumerate}
\item Zeige: Wenn die Kongruenz~$x^2 \equiv d$ mod~$p$ unlösbar ist,
dann ist~$(p) \subseteq \O_K$ ein Primideal.
\item Zeige, dass auch die Umkehrung der Aussage aus~a) gilt, falls~$p$
ungerade ist.
\end{enumerate}
{\tiny\emph{Hinweis.} Ist~$f \in \FF_p[X]$ ein quadratisches Polynom, so ist
der Ring~$\FF_p[X]/(f)$ genau dann ein Integritätsbereich, wenn~$f$ keine
Nullstelle modulo~$p$ besitzt.\par}
\end{aufgabe}

\begin{aufgabe}{Eine Einschränkung an Diskriminanten von Zahlkörpern}
Sei~$d$ die Diskrimante einer~$\QQ$-Basis eines beliebigen Zahlkörpers, welche
nur aus ganzen Elementen besteht. Zeige, dass~$d$ modulo~$4$ gleich~$0$ oder~$1$ ist.

{\tiny\emph{Hinweis.} Solltest du in die Situation kommen, Körpereinbettungen
auf gewisse komplexe Zahlen anzuwenden, dann stress dich nicht, falls diese
Zahlen gar nicht in der Definitionsmenge der Körpereinbettungen liegen sollten.
Das ist ein behebbares technisches Problem.\par}
\end{aufgabe}

\begin{aufgabe}{Ein Konstruktionsverfahren für Ganzheitsbasen}
Sei~$K$ ein Zahlkörper und~$(\vartheta_1,\ldots,\vartheta_n)$ eine~$\QQ$-Basis
von~$K$, welche nur aus ganzen Elementen besteht. Sei~$d$ ihre Diskriminante.
Für~$i = 1, \ldots, n$ wählen wir aus der Menge
\[ B_i \defeq \{ x \in \O_K \,|\,
  \text{$x = \tfrac{1}{d}(a_1\vartheta_1 + a_2\vartheta_2 + \cdots +
  a_i\vartheta_i)$ für gewisse~$a_1,\ldots,a_i \in \ZZ$ mit~$a_i
  \neq 0$} \} \]
ein Element~$x_i$ mit minimalem Betrag des Koeffizienten~$a_i$. Zeige,
dass~$(x_1,\ldots,x_n)$ eine Ganzheitsbasis von~$\O_K$ ist.
\end{aufgabe}

\begin{aufgabe}{Wir suchen eine Ganzheitsbasis}
Sei~$\alpha \in \CC$ eine Nullstelle des Polynoms~$X^3 - X - 4 \in \QQ[X]$.
Sei~$K \defeq \QQ[\alpha]$. Finde eine Ganzheitsbasis von~$\O_K$.

{\tiny\emph{Hinweis.} Verwende etwa das Verfahren aus Aufgabe 4. Zur Kontrolle:
Die Diskriminante von~$(1,\alpha,\alpha^2)$ ist~$-2^2 \cdot 107$. Wenn du
überprüfen möchtest, ob ein Element aus~$K$ ganz ist, dann berechne zunächst
seine Spur und dann seine Norm; sind diese nicht ganzzahlig, so kann das
Element nicht ganz sein. Im schlimmsten Fall musst du folgende
Charakterisierung verwenden (in dieser Aufgabe kommt man aber darum herum): Ein
Element~$u$ aus~$K$ ist genau dann ganz, wenn das charakteristische Polynom der
linearen Abbildung~$K \to K,\,x \mapsto ux$ ganzzahlige Koeffizienten hat.\par}
\end{aufgabe}

\begin{aufgabe*}{Der mystische Körper mit einem Element}
Das~\emph{$q$-Analogon} einer Zahl~$n$ ist~$[n]_q \defeq 1+q+\cdots+q^{n-1}$.
\begin{enumerate}
\item Zeige: Ein~$n$-dimensionaler Vektorraum über~$\FF_q$ besitzt
genau~$(q^n-1)/(q-1)$ eindimensionale Untervektorräume.
\item Finde eine Formel für die Anzahl~$k$-dimensionaler Untervektorräume
eines~$n$-dimensionalen Vektorraums über~$\FF_q$. Drücke dein Ergebnis
über~$q$-Analoga aus.
\item Setze ohne Erlaubnis in deiner Formel aus~b)~$q \defeq 1$. Was passiert?
Was sollten also Vektorräume und Untervektorräume über dem
Körper~$\FF_1$ mit einem Element sein?
\end{enumerate}
\end{aufgabe*}

\end{document}
