\documentclass{uebblatt}

\usepackage{framed}

\definecolor{haskell}{RGB}{150,0,255}

\newcommand{\hilight}[2]{\begin{center}%
  \setlength{\fboxrule}{2pt}%
  \setlength{\fboxsep}{5pt}%
  \textcolor{haskell}{\fbox{\textcolor{black}{\parbox{#1}{#2}}}}\end{center}}

\newcommand{\Cl}{\mathrm{Cl}}
\newcommand{\Min}{\mathrm{Min}}

\let\raggedsection\centering

\begin{document}

\section*{Von Minkowski zur Endlichkeit der Idealklassengruppe}

Sei~$K$ ein Zahlkörper vom Grad~$n$ mit~$s$ Pärchen komplexer Einbettungen. Wir
wollen verstehen, wieso die Idealklassengruppe~$\Cl_K$ endlich ist. Genauer:
wieso es endlich viele Ideale~$\aaa_1, \ldots, \aaa_m \subseteq \O_K$ mit
\[ \Cl_K = \{ [\aaa_1], \ldots, [\aaa_m] \} \]
gibt -- und wie man diese Ideale bestimmen kann.
Grundlegend dazu ist folgendes Resultat:

\hilight{0.75\textwidth}{Zu jedem Element~$g \in \Cl_K$ gibt es ein
Ideal~$\aaa \subseteq \O_K$ mit~$g = [\aaa]$ und
\[ N(\aaa) \leq \left(\frac{4}{\pi}\right)^s \cdot \frac{n!}{n^n} \cdot
\sqrt{|d_K|} =\vcentcolon \Min_K. \]\vspace*{-1.3em}}

Denn nach Definition gilt ja zunächst nur
\begin{align*}
  \Cl_K &= \{ [\aaa] \,|\, \text{$\aaa \subseteq K$ gebrochenes Ideal} \}. \\
\intertext{Wegen Minkowskis Resultat kann man aber auch}
  \Cl_K &= \{ [\aaa] \,|\, \text{$\aaa \subseteq \O_K$ Ideal mit~$\aaa \neq
  (0)$ und~$N(\aaa) \leq \Min_K$} \}
\end{align*}
schreiben. Das ist eine starke Einschränkung, denn -- wie wir gleich sehen
werden -- gibt es von diesen Idealen nur endlich viele; und mehr noch: Man kann
sie finden und explizit angeben.

Dazu betrachten wir für den Moment ein beliebiges Ideal~$\aaa \subseteq \O_K$
mit~$\aaa \neq (0)$ und~$N(\aaa) \leq \Min_K$. Welche Primideale können in der
Primidealzerlegung von~$\aaa$ nur vorkommen? Wenn wir~$\aaa = \ppp_1^{\nu_1}
\cdots \ppp_k^{\nu_k}$ schreiben, so gilt
\[ N(\ppp_i) \leq N(\ppp_i)^{\nu_i} = N(\ppp_i^{\nu_i}) \leq
  N(\ppp_1^{\nu_1}) \cdots N(\ppp_k^{\nu_k}) = N(\ppp_1^{\nu_1}
  \cdots \ppp_k^{\nu_k}) = N(\aaa) \leq \Min_K. \]
Ferner erkennen wir: Das Primideal~$\ppp_i$ ist nicht irgendein Primideal.
Vielmehr ist es eines der Faktoren in der Primidealzerlegung von~$(p) \subseteq
\O_K$, wobei~$p \in \ZZ$ die Primzahl mit~$(p) = \ppp_i \cap \ZZ \subseteq \ZZ$
ist.\footnote{Weiter vorne im Satz ist mit~"`$(p)$"' das von~$p$ erzeugte Ideal
von~$\O_K$ gemeint; weiter hinten das von~$p$ erzeugte Ideal in~$\ZZ$. Die
Behauptung kann man in drei Schritten einsehen:
\begin{enumerate}
\item[1.] Dass es überhaupt eine Primzahl~$p \in \ZZ$ gibt, für die~$(p) = \ppp_i
\cap \ZZ$ ist, liegt daran, dass aus ganz allgemeinen ringtheoretischen Gründen
das Ideal~$\ppp_i \cap \ZZ$ ein Primideal von~$\ZZ$ ist und dass dieses
Primideal nicht das Nullideal ist (das liegt an der Ganzheit von~$\O_K$
über~$\ZZ$).
\item[2.] Aus allgemeinen ringtheoretischen Gründen folgt~$(p) \subseteq \ppp_i$.
("`Idealerweiterung ist linksadjungiert zu Idealkontraktion."')
\item[3.] Sei~$(p) = \qqq_1 \cdots \qqq_m \subseteq \O_K$ die Primidealzerlegung
von~$(p)$. Da~$\ppp_i$ ein Primideal ist, folgt aus~$\qqq_1 \cdots \qqq_m
\subseteq \ppp_i$ schon, dass es einen Index~$j$ mit~$\qqq_j \subseteq \ppp_i$
gibt. Da~$\qqq_j$ wie jedes nichttriviale Primideal in einem Dedekindring
maximal ist, folgt~$\qqq_j = \ppp_i$.
\end{enumerate}} Diese Primzahl ist nicht beliebig groß, denn es gilt
\[ p \leq p^{f_i} = N(\ppp_i) \leq \Min_K, \]
wenn man mit~"`$f_i$"' die endliche Dimension
des~$\FF_p$-Vektorraums~$\O_K/\ppp_i$ bezeichnet.

Das Fazit dieser Überlegung lautet:\enlargethispage{1em}

\hilight{0.88\textwidth}{Seien~$\ppp_1, \ldots, \ppp_\ell$
die endlich vielen Primideale, die in den Primidealzerlegungen der endlich
vielen Primideale~$(p) \subseteq \O_K$, wobei~$p \in \ZZ$ über alle
Primzahlen mit~$p \leq \Min_K$ läuft, vorkommen. Dann gilt
\[ \Cl_K = \{ [\aaa] \,|\, \text{$\aaa \subseteq \O_K$ ist ein Produkt
der~$\ppp_1,\ldots,\ppp_\ell$ mit~$N(\aaa) \leq
\Min_K$} \}. \]\vspace*{-1.8em}}
Die~$\ppp_i$ dürfen in diesen Produkten durchaus mit Vielfachheit Null oder
auch mit Vielfachheit größer als Eins auftreten.


\newpage
\section*{Ein Beispiel: Die Klassengruppe von~$\QQ(\sqrt{-5})$}

Sei~$K = \QQ(\sqrt{-5})$. Da~$-5 \equiv 3$ modulo~$4$, ist die Diskriminante
von~$K$ gleich~$4 \cdot (-5)$ und die Minkowskischranke daher
\[ \Min_K = \left(\frac{4}{\pi}\right)^1 \cdot \frac{2!}{2^2} \cdot \sqrt{20}
  \approx 2{,}85. \]
Somit müssen wir nur die Primzahl~$p = 2$ untersuchen, um alle Elemente der
Klassengruppe auflisten zu können.

Der Führer~$\FFF_{\sqrt{-5}}$ ist das Einsideal, denn~$\O_K = \ZZ[\sqrt{-5}]$.
Daher genügt es, um die Primidealzerlegung von~$(2) \subseteq \O_K$ zu
bestimmen, die Zerlegung des Minimalpolynoms~$X^2 + 5$ über~$\FF_2$ zu
bestimmen. Diese lautet
$X^2 + 5 = (X + 1)^2 \in \FF_2[X]$.
Daher zerlegt sich das Ideal~$(2)$ als
\[ \text{$(2) = \ppp^2$ mit~$\ppp = (2, \sqrt{-5} + 1)$. } \]
Somit folgt
$\Cl_K = \{ [\ppp^\nu] \,|\, \text{$\nu \geq 0$ mit~$N(\ppp^\nu) \leq
\Min_K$} \}$.
Das können wir noch aufdröseln: Es gilt~$N(\ppp) = |\O_K/\ppp| = |\FF_2| = 2$,
also~$N(\ppp^\nu) = 2^\nu$. Für~$\nu$ sind daher nur die Werte~$0$ und~$1$
möglich. Somit besteht die Klassengruppe aus höchstens zwei
Elementen:
\[ \Cl_K = \{ [(1)], [\ppp] \}. \]
Ferner ist~$\ppp$ kein Hauptideal, denn eine Nebenrechnung zeigt, dass die Norm
eines Hauptideals stets von der Form~$a^2 + 5b^2$ mit~$a,b \in \ZZ$ ist. Daher
ist~$[\ppp] \neq [(1)]$; die Klassengruppe besteht aus genau zwei Elementen.


\section*{Eine Warnung}

Man könnte denken, dass die Abschätzung~$h_K \leq \Min_K$ gilt. Das ist jedoch
falsch. Im Fall~$K = \QQ(\sqrt{-71})$ ist~$h_K = 7$ und~$\Min_K =
2\sqrt{71}/\pi \approx 5{,}36$.

\end{document}
