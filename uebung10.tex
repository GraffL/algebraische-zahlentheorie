\documentclass[entwurf]{uebblatt}
\begin{document}

\maketitle{10}

\begin{aufgabe}{Das inverse galoissche Problem im abelschen Fall}
Sei~$A$ eine endliche abelsche Gruppe. Konstruiere einen Zahlkörper~$K$
mit~$\Gal(K|\QQ) \cong A$.

{\tiny\emph{Hinweis.} Wir können~$A \cong \ZZ/(n_1) \times \cdots \times
\ZZ/(n_r)$ schreiben und nach Dirichlets Satz \emph{verschiedene}
Primzahlen~$p_i$ mit~$p_1 \equiv 1$ modulo~$n_i$ finden. Wir können dann den
gesuchten Zahlkörper~$K$ als den Fixkörper
des Körpers~$\QQ(\zeta_{p_1}\cdots\zeta_{p_r})$ bezüglich einer geeigneten Untergruppe
seiner Galoisgruppe finden.\par}
\end{aufgabe}

\end{document}
