\documentclass{uebblatt}
\begin{document}

\maketitle{13}

\begin{aufgabe}{Triviales zu Bewertungen}
Sei~$|\cdot|$ eine Bewertung auf einem Körper~$K$.
\begin{enumerate}
\item Zeige, dass~$|\cdot|$ genau dann die verschärfte Dreiecksungleichung~$|x
+ y| \leq \max\{|x|,|y|\}$ erfüllt, wenn für alle~$x \in K$ aus~$|x| \leq 1$
folgt, dass~$|x + 1| \leq 1$.
\end{enumerate}
Gelte von nun an die verschärfte Dreiecksungleichung.
\begin{enumerate}
\addtocounter{enumi}{1}
\item Seien~$x,y \in K$ mit~$|x| \neq |y|$. Zeige, dass~$|x + y| = \max\{ |x|, |y| \}$.
\item Zeige, dass alle Dreiecke in~$K$ gleichschenklig sind.
\end{enumerate}
{\tiny\emph{Tipp.} Dividiere durch~$|x|$ oder~$|y|$. Schreibe sowas wie~"`$|x +
(y - x)|$"'.\par}
\end{aufgabe}

\begin{aufgabe}{Triviale Bewertungen}
Sei~$L|K$ eine algebraische Körpererweiterung. Sei~$w$ eine
Exponentialbewertung auf~$L$, welche auf~$K$ trivial ist (d.\,h.~$w(x) = 0$ für
alle~$x \in K^\times$). Zeige, dass~$w$ auf~$L$ trivial ist.
\end{aufgabe}

\begin{aufgabe}{Charakterisierung nichtarchimedischer Bewertungen}
Sei~$|\cdot|$ eine nichtarchimedische Bewertung auf einem Zahlkörper~$K$.
\begin{enumerate}
\item Sei zunächst~$K = \QQ$. Zeige, dass es eine Primzahl~$p$ mit~$|\cdot| =
|\cdot|_p$ gibt.
\item Zeige, dass es ein Primideal~$\ppp \subseteq \O_K$ mit~$\ppp \neq (0)$
und~$|\cdot| = |\cdot|_\ppp$ gibt.
\end{enumerate}
{\tiny\emph{Hinweis.} Die Bewertung erfüllt die verschärfte
Dreiecksungleichung. Zeige zunächst, dass~$|x| \leq 1$ für alle~$x \in \ZZ$.
Zeige dann, dass~$\{ x \in \ZZ \,|\, |x| < 1 \}$ ein nichttriviales Primideal
von~$\ZZ$ ist. Es ist also von der Form~$(p)$ für eine Primzahl~$p$. Für diese
Primzahl~$p$ kannst die Behauptung nachweisen. Der Beweis im allgemeinen Fall
verläuft analog, mit~$\O_K$ statt~$\ZZ$.\par}
\end{aufgabe}

\begin{aufgabe}{Bewertung irreduzibler Polynome}
Sei~$K$ ein vollständig diskret bewerteter Körper. Sei~$\O \defeq \{ x \in K
\,|\, |x| \leq 1 \}$ sein Bewertungsring. Sei~$f(X) = a_n X^n + \cdots
+ a_1 X + a_0 \in K[X]$ ein irreduzibles Polynom.
\begin{enumerate}
\item Zeige, dass~$|f(X)| = \max\{|a_0|,|a_n|\}$.

{\tiny\emph{Hinweis.} Nach Definition ist~$|f(X)| =
\max\{|a_0|,\ldots,|a_n|\}$. Verwende Hensels Lemma in seiner allgemeinen
Formulierung.\par}
\item Folgere: Ist~$f(X)$ normiert und~$a_0 \in \O$, so gilt schon~$f(X) \in
\O[X]$.
\end{enumerate}
\end{aufgabe}

\begin{aufgabe}{Fortsetzung von Bewertungen}
Sei~$K$ ein vollständig diskret bewerteter Körper~$K$. Sei~$L|K$ eine
Erweiterung vom Grad~$n$. Zeige, dass die Setzung~$|x| \defeq
\sqrt[n]{|N_{L|K}(x)|}$ für~$x \in L$ eine Bewertung auf~$L$ definiert, welche
die gegebene Bewertung auf~$K$ fortsetzt.

{\tiny\emph{Tipp.} Zeige zunächst, dass der ganze Abschluss des
Bewertungsrings~$\O_K$ von~$K$ in~$L$ gleich~$\{ x \in L \,|\, N_{L|K}(x) \in
\O_K \}$ ist. Verwende dazu die bekannte Formel, die Norm und den konstanten
Koeffizienten des Minimalpolynoms miteinander in Beziehung setzt. Die
Inklusion~"`$\subseteq$"' wurde schon vor langer Zeit behandelt. Nutze für die
andere Inklusion die Folgerung aus der vorherigen Teilaufgabe.\par}
\end{aufgabe}

\end{document}
Zeige, dass im Fall eines vollständig diskret bewerteten Körpers K die Formel
|x|=\sqrt[n]{ |
N_{L|K}(x) |} für eine endliche Erweiterung L|K eine Bewertung definiert, die
|.| auf K fortsetzt.
(Hinweis: Zeige Folgendes: ist O der Bewertungsring von K und O_L dessen ganzer
Abschluss in L, so
gilt O_L=\{ x \in L \mid N_{L|K}(x) \in O \} — Aufgabe zuvor nutzen!).
(noch ein Hinweis: zeige, dass äquivalent sind:
1) |x+y|\le max(|x|,|y|) für alle x,y \neq 0
2) |x| \le 1 \Leftrightarrow |x+1| \le 1).
