\documentclass{uebblatt}

\begin{document}

\maketitle{2}

\begin{aufgabe}{Zu erwartende und überraschende Ganzheit}
\begin{enumerate}
\item Zeige, dass~$\sqrt{2}$ ganz über~$\ZZ$ ist.
\item Zeige, dass~$\tfrac{1}{\sqrt{2}}$ nicht ganz über~$\ZZ$ ist.
\item Zeige, dass~$\tfrac{1+\sqrt{-7}}{2}$ ganz über~$\ZZ$ ist.
\end{enumerate}
\end{aufgabe}

\begin{aufgabe}{Produkt ganzer Zahlen}
\begin{enumerate}
\item Seien~$x$ und~$y$ komplexe Zahlen mit~$x^3-x+1=0$ und~$y^2-2=0$.
Finde eine normierte Polynomgleichung mit ganzzahligen
Koeffizienten, die die Zahl~$x \cdot y$ als Lösung besitzt.
\item Der \emph{Grad} einer ganzen Zahl~$z$ ist der kleinstmögliche Grad
einer normierten Polynomgleichung mit ganzzahligen Koeffizienten, die~$z$ als Lösung
besitzt. Finde eine Abschätzung für den Grad des Produkts zweier ganzer Zahlen
in Abhängigkeit der Grade der Faktoren.
\end{enumerate}
\end{aufgabe}

\begin{aufgabe}{Erste Schritte mit Norm und Spur}
Sei~$\alpha \in \CC$ eine Nullstelle des Polynoms~$f(X) \defeq X^3 - 2\,X + 5$.
Sei~$K \defeq \QQ[\alpha]$. Berechne Norm und Spur des Elements~$2\alpha - 1
\in K$:
\begin{enumerate}
\item Begründe kurz, wieso~$(1,\alpha,\alpha^2)$ eine~$\QQ$-Basis von~$K$ ist.
\item Stelle eine Darstellungsmatrix der linearen Abbildung~$K \to K,\,z \mapsto
(2\alpha-1)z$ auf.
\item Berechne Determinante und Spur dieser Matrix.
\end{enumerate}
\end{aufgabe}

\begin{aufgabe}{Knobeln mit Einheitswurzeln}
Sei~$p \in \NN$ eine Primzahl und~$\zeta \in \CC$ eine primitive~$p$-te
Einheitswurzel (also~$\zeta^p = 1$ und~$\zeta \neq 1$). Sei~$K \defeq
\QQ[\zeta]$.
\begin{enumerate}
\item Was ist die Norm von~$1 - \zeta \in K$?
{\tiny\emph{Hinweis.} Das Minimalpolynom von~$\zeta$ ist~$X^{p-1} + X^{p-2} +
\cdots + X + 1 = \tfrac{X^p - 1}{X-1}$.\par}
\item Sei~$k \in \NN$. Zeige, dass die Zahl~$\gamma_k \defeq
\tfrac{1-\zeta^k}{1-\zeta} \in K$ in~$\O_K$ liegt.
\item Seien nun~$k$ und~$p$ zueinander teilerfremd. Zeige, dass~$\gamma_k$
sogar in~$\O_K^\times$ liegt.
\end{enumerate}
\end{aufgabe}

\begin{aufgabe*}{Einheitswurzeln in nichtkommutativen Matrixringen}
Bestimme die Anzahl der~$(4\times4)$-Matrizen~$A$ mit rationalen Einträgen
und~$A^7 = I$, $A \neq I$.
\end{aufgabe*}

\end{document}
