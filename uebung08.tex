\documentclass{uebblatt}
\begin{document}

\maketitle{8}

\begin{aufgabe}{Verzweigung ist die Ausnahme}
Sei~$K$ ein Zahlkörper vom Grad~$n$. Gelte~$K = \QQ[\vartheta]$ mit~$\vartheta
\in \O_K$.
\begin{enumerate}
\item Zeige: Die Diskriminante~$d_\vartheta$
der~$\QQ$-Basis~$(1,\vartheta,\ldots,\vartheta^{n-1})$ von~$K$ ist gleich der
Diskriminante des Minimalpolynoms von~$\vartheta$.
\item Sei~$p$ eine Primzahl, sodass die Ideale~$(p)$ und~$\FFF_\vartheta$
von~$\O_K$ zueinander teilerfremd sind. Zeige, dass~$p$ genau dann in~$K$
verzweigt ist, wenn~$p \mid d_\vartheta$.
\item Zeige: Nur endlich viele Primzahlen sind in~$K$ verzweigt. Kannst du die
Kandidaten für verzweigte Primzahlen sogar explizit angeben? Was ist die
Konsequenz für die Visualisierung von Ganzheitsringen im Stile von Mumfords
Schatzkarte?
\item Interpretiere Aufgabe 2 von Blatt 4 in neuem Licht.
\end{enumerate}
\end{aufgabe}

\begin{aufgabe}{Trägheit bei nichtzyklischer Galoisgruppe}
Sei~$L|K$ eine Galoiserweiterung von Zahlkörpern. Sei~$\Gal(L|K)$ nicht zyklisch.
\begin{enumerate}
\item Zeige, dass kein Primideal von~$\O_K$ in~$L$ träge ist.

{\tiny\emph{Tipp.} Verwende ohne Beweis, dass für Primideale~$\PPP$ über~$\ppp$
mit Verzweigungsindex~1 gilt, dass~$G_\PPP \cong
\Gal(\O_L/\PPP|\O_K/\ppp)$.\par}
\item Folgere, dass nur endlich viele Primideale von~$\O_K$ in~$L$ unzerlegt sind.

{\tiny\emph{Hinweis.} Ein Scholium von Aufgabe~1c) ist, dass nur endlich viele
Primideale von~$\O_K$ in~$L$ verzweigt sind.\par}
\end{enumerate}
\end{aufgabe}

\begin{aufgabe}{Vorfreude aufs quadratische Reziprozitätsgesetz, Gauß' aureum theorema}
\begin{enumerate}
\item Ist~$10$ modulo~$p \defeq 65537$ ein quadratischer Rest?

{\tiny\emph{Hinweis.} Verwende ohne Beweis, dass $\leg{10}{p} = \leg{2}{p}
\leg{5}{p}$ (das ist nicht krass) und~$\leg{5}{p} = \leg{p}{5}$ (das ist krass).\par}

\item Bestimme die Periodenlänge der Dezimalbruchentwicklung von~$1/65537$.

%{\tiny\emph{Hinweis.} Die Periodenlänge ist die Ordnung von~$10$ in der
%multiplikativen Gruppe~$(\ZZ/(65537))^\times$ (wieso?). Als solche teilt
%sie~$2^{16}$ (wieso?). Sie ist genau dann ein echter Teiler von~$2^{16}$,
%wenn~$10$ modulo~$65537$ ein quadratischer Rest ist (wieso?).\par}
\end{enumerate}
\end{aufgabe}

\begin{aufgabe*}{Lücken zwischen Primzahlen}
Zeige: Zu jeder Lauflänge~$n \geq 1$ gibt es eine Folge von~$n$
aufeinanderfolgenden natürlichen Zahlen, welche alle keine Primzahlen sind.
\end{aufgabe*}

\begin{aufgabe*}{Verzweigte Überlagerungen in der komplexen Geometrie}
\begin{enumerate}
\item Informiere dich über verzweigte Überlagerungen (branched coverings) in der
komplexen Geometrie und vergleiche die dortige Situation mit der fundamentalen
Gleichung.
\item Frage Sven, was er dir zu diesem Thema auf jeden Fall mitgeben möchte.
\end{enumerate}
\end{aufgabe*}

\end{document}

http://www.math.uni-hamburg.de/home/posingies/aufgaben/aztlsg5.pdf

https://web.archive.org/web/20060903142509/http://modular.math.washington.edu/129-05/notes/129.pdf
Aufgabe zu quadratischer Reziprozität
