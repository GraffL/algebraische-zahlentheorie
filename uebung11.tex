\documentclass{uebblatt}
\begin{document}

\maketitle{11}

\begin{aufgabe}{Endlichkeit der Untergruppe der Einheitswurzeln}
\begin{enumerate}
\item Sei~$K$ ein Zahlkörper. Zeige, dass~$\mu(K) = \{ \zeta \in K \,|\,
\text{$\zeta^k = 1$ für ein~$k \in \NN$} \}$ endlich ist.

{\tiny\emph{Erinnerung.} Es gilt~$\varphi(ab) = \varphi(a) \varphi(b)$ für
teilerfremde Zahlen~$a$ und~$b$ und~$\varphi(p^r) = p^r - p^{r-1}$.\par}
%{\tiny\emph{Tipp.} Weise nach, dass es zu jeder Zahl~$N \geq 1$ nur endlich
%viele Zahlen~$k$ mit~$\varphi(k) = N$ gibt.\par}

\item Sei~$K$ ein Zahlkörper, der über eine Einbettung nach~$\RR$ verfügt.
Zeige: $\mu(K) = \{ \pm1 \}$.

\item Für welche Zahlkörper enthält der zugehörige Ganzheitsring nur endlich
viele Einheiten? Was ist eine obere Schranke dafür, wie viele Einheiten diese
Ringe höchstens enthalten?
\end{enumerate}
\end{aufgabe}

\begin{aufgabe}{Die Schlacht an der Bucht der Drachen}
\begin{enumerate}
\item Sei~$K \defeq \QQ[\sqrt{13}]$. Bestimme eine fundamentale Einheit
von~$\O_K$, also ein Einheit~$\varepsilon$, sodass jede Einheit von~$\O_K$ von der
Form~$\pm \varepsilon^m$ für ein~$m \in \ZZ$ ist.

{\tiny\emph{Tipp.} Du weißt, wie~$\O_K$ aussieht. Zeichne das
Gitter~$(\ell \circ j)[\O_K^\times]$ in~$H = \{ (x,y) \in \RR^2 \,|\, x + y = 0
\} \subseteq \RR^2$ und
finde einen Punkt, der dem Ursprung am nächsten ist. Das ist eine fundamentale
Einheit.\par}

\item Als sich Daenerys' Armee der Unbefleckten im üblichen militärischen Stil
formiert hatte, sahen ihre Gegner 13 gleich große quadratische Einheiten.
Zusammen mit Daenerys selbst hätten sie aber auch ein einzelnes großes Quadrat
bilden können. Wie viele Krieger umfasste die Armee und wie alt ist der
Busfahrer?

{\tiny\emph{Tipp.} Für den ersten Teil~Aufgabe a).\par}
\end{enumerate}
\end{aufgabe}

\begin{aufgabe}{Ein allgemeines Beispiel zur Einheitenbestimmung}
\begin{enumerate}
\item Sei~$K$ ein Zahlkörper mit~$r$ reellen und~$s$ Paaren komplexer
Einbettungen. Zeige, dass das Vorzeichen der Diskriminante von~$K$
gleich~$(-1)^s$ ist.
\item Sei~$K$ ein Zahlkörper vom Grad~3 mit negativer Diskriminante. Zeige,
dass es eine fundamentale Einheit~$\varepsilon \in \O_K$ gibt und dass~$K =
\QQ(\varepsilon)$ ist.

{\tiny\emph{Tipp.} Weise zunächst nach, dass~$K$ über genau eine reelle und
genau ein Pärchen komplexer Einbettungen verfügt. Damit ausgestattet liefert
dir der Dirichletsche Einheitensatz eine fundamentale Einheit~$\varepsilon$.
Weise nun nach, dass~$[\QQ(\varepsilon) : \QQ] = [K : \QQ] = 3$. Damit folgt~$K
= \QQ(\varepsilon)$.\par}
\end{enumerate}
\end{aufgabe}

\begin{aufgabe}{Einheiten von rein reellen Zahlkörpern}
Sei~$K$ ein rein reeller Zahlkörper (d.\,h. dass das Bild einer jeden
Körpereinbettung~$K \to \CC$ schon in~$\RR$ liegt). Sei~$X$ eine echte
nichtleere Teilmenge von~$\Hom(K, \RR)$. Zeige, dass es eine
Einheit~$\varepsilon \in \O_K$ mit~$0 < \sigma(\varepsilon) < 1$ für alle~$\sigma \in X$
und~$\sigma(\varepsilon) > 1$ für alle~$\sigma \not\in X$ gibt.

{\tiny\emph{Hinweis.} Verwende den Gitterpunktsatz von Minkowski für das
Einheitengitter in der Spur-Null-Hyperebene.\par}
\end{aufgabe}

\end{document}

sortBy (\(_,_,v) (_,_,v') -> compare v v') $ let ns = [-100..100] in [ (x,y, sqrt ((log (abs $ toNum x y))^2 + (log (abs $ toNum x (-y))^2))) | x <- ns, y <- ns, x^2 - 13*y^2 == -4 || x^2 - 13*y^2 == 4 ]
[(-2,0,0.0),(2,0,0.0),(-3,-1,1.6896503457119434),(-3,1,1.6896503457119434),(3,-1,1.6896503457119434),(3,1,1.6896503457119434),(-11,-3,3.3793006914238823),(-11,3,3.3793006914238823),(11,-3,3.3793006914238823),(11,3,3.3793006914238823),(-36,-10,5.0689510371358395),(-36,10,5.0689510371358395),(36,-10,5.0689510371358395),(36,10,5.0689510371358395)]
