\documentclass{uebblatt}

\begin{document}

\maketitle{6}

\begin{aufgabe}{Klassenzahlberechnungen}
\begin{enumerate}
\item Zeige, dass die quadratischen Zahlkörper~$\QQ[\sqrt{d}]$ für~$d \in \{
-7, -3, -2, -1, 2, 3, 5 \}$ die Klassenzahl~$1$ besitzen.
\item Zeige, dass auch~$\QQ[\sqrt{7}]$ die Klassenzahl~$1$ besitzt.
\item Was ist die Klassenzahl von~$\QQ[\sqrt{-5}]$?
\end{enumerate}
\end{aufgabe}

\begin{aufgabe}{Eine Schranke für die Diskriminante}
\begin{enumerate}
\item Sei~$K$ ein Zahlkörper vom Grad~$n$. Sei~$d_K$ die Diskriminante einer
Ganzheitsbasis. Zeige:
\[ |d_K| \geq (n^n / n!)^2 \cdot (\pi/4)^n. \]
\item Zeige: Bis auf~$\QQ$ selbst gibt es keinen Zahlkörper mit~$|d_K| = 1$.
\end{enumerate}
\end{aufgabe}

\begin{aufgabe}{Wir mögen Hauptideale}
\begin{enumerate}
\item Sei~$K$ ein Zahlkörper und~$\aaa \subseteq \O_K$ ein Ideal. Sei~$\aaa^m =
(\alpha)$ für ein~$\alpha \in \O_K$ und eine Zahl~$m \geq 0$. Sei~$L \defeq
K(\sqrt[m]{\alpha})$. Zeige, dass das Ideal~$\aaa \O_L$ von~$\O_L$ ein
Hauptideal ist.

{\tiny\emph{Hinweis.} Nur um Missverständnissen vorzubeugen, das
Ideal~$\aaa\O_L$ besteht aus allen~$\O_L$-Linearkombinationen von Elementen
aus~$\aaa$.\par}
\item Sei~$K$ ein Zahlkörper. Finde eine endliche Erweiterung~$L$ von~$K$,
sodass jedes Ideal von~$\O_K$ in~$\O_L$ zu einem Hauptideal wird (im gleichen
Sinn wie in~a)).

{\tiny\emph{Hinweis.} Versuche, "`die Klassengruppe zu töten"'.\par}
\end{enumerate}
\end{aufgabe}

\end{document}

-- für Aufgabe 1b: Wieso ist (2, 1 + sqrt(7)) ein Hauptideal?
> let xs = [-6..6] in [ (a,b,c,d,e,f) | a <- xs, b <- xs, c <- xs, d <- xs, e <- xs, f <- xs, 2*a*c+b*c+7*b*d == 2, 2*a*d+b*d+b*c == 0, 2*a*e+b*e+7*b*f == 1, 2*a*f+b*f+b*e == 1 ]
[(-2,1,-3,-1,-5,-2),(-1,-1,-3,1,2,-1),(1,1,3,-1,-2,1),(2,-1,3,1,5,2)]
