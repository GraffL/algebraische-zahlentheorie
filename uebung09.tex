\documentclass[entwurf]{uebblatt}
\begin{document}

\maketitle{9}

\begin{aufgabe}{Ein Geheimnis der Zahl 5}
\begin{enumerate}
\item Sei~$x \in \ZZ$. Sei~$p$ eine Primzahl. Zeige: $\leg{x}{p} \equiv
x^{(p-1)/2}$ modulo~$p$.
\item Sei~$p$ eine Primzahl. Sei~$F_p$ die~$p$-te Fibonaccizahl. Zeige: $F_p
\equiv \leg{5}{p}$ modulo~$p$.

{\tiny\emph{Tipp.} Verwende die bekannte Formel~$F_n = (\Phi^n - \Psi^n) /
(\Phi - \Psi)$, wobei~$\Phi = (1 + \sqrt{5})/2$ und~$\Psi = (1 - \sqrt{5})/2$.
Zwei ganze Zahlen teilen einander genau dann, wenn sie es
in~$\O_{\QQ[\sqrt{5}]}$ tun (wieso?).\par}
\end{enumerate}
\end{aufgabe}

\end{document}
