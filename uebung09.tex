\documentclass{uebblatt}
\begin{document}

\maketitle{9}

\begin{aufgabe}{Beispiele für interessantes Verzweigungsverhalten}
Finde Beispiele für Galoiserweiterungen~$L|K$ von Zahlkörpern und
Primideale~$\ppp \subseteq \O_K$ mit~$\ppp \neq (0)$, für die
\begin{enumerate}
\item in der Primidealzerlegung von~$\ppp\O_L$ mindestens zwei Ideale vorkommen
("`$r \geq 2$"'),
\item die Primfaktoren mindestens Verzweigungsindex zwei haben ("`$e \geq 2$"'),
\item der gemeinsame Trägheitsgrad der Primfaktoren mindestens zwei ist ("`$f \geq 2$"').
\end{enumerate}
{\tiny\emph{Präzisierung.} Finde drei einzelne Beispiele oder Beispiele, die
mehrere der Wünsche erfüllen. Ganz wie du willst.\par}
\end{aufgabe}

\begin{aufgabe}{Faktorisierung in Zerlegungskörper und Trägheitskörper}
\marginpar{\ \\\ \\{\qquad}$\xymatrix{
  L \ar@{-}[d]^e \\
  T_\PPP \ar@{-}[d]^f \\
  Z_\PPP \ar@{-}[d]^r \\
  K
}$}
Sei~$L|K$ eine Galoiserweiterung von Zahlkörpern. Sei~$\PPP \subseteq \O_L$ ein
Primideal mit~$\PPP \neq (0)$. Sei~$e \defeq e(\PPP|\ppp)$ und~$f \defeq f(\PPP|\ppp)$.
Zeige, dass wir den nebenstehenden Körperturm haben.
\begin{enumerate}
\item Wieso liegt~$Z_\PPP$ in~$T_\PPP$?
\item Wieso ist~$[L : Z_\PPP] = ef$?
\item Wieso ist~$T_\PPP$ über~$Z_\PPP$ normal und wieso ist~$\Gal(T_\PPP|Z_\PPP) \cong \Gal(\kappa(\PPP)|\kappa(\ppp))$?
\item Wieso ist~$[L : T_\PPP] = e$ und wieso ist~$[T_\PPP : Z_\PPP] = f$?
\item[$\heartsuit$ e)] Sei~$\rrr \defeq \PPP \cap \O_{T_\PPP}$.
Sei~$\qqq \defeq \PPP \cap \O_{Z_\PPP}$. Zeige~$\kappa(\rrr) = \kappa(\PPP)$ und folgere:
\[ e(\PPP|\rrr) = e(\PPP|\ppp), \quad
  f(\PPP|\rrr) = 1, \quad
  e(\rrr|\qqq) = 1, \quad
  f(\rrr|\qqq) = f. \]
\end{enumerate}
\end{aufgabe}

\begin{aufgabe}{Ein Spezialfall von Dirichlets Satz über Primzahlen in arithmetischen
Progressionen}
Sei~$n$ eine positive natürliche Zahl. Sei~$\Phi_n$ das~$n$-te
Kreisteilungspolynom. Seien~$p_1,\ldots,p_r$ Primzahlen
mit~$p_i \equiv 1$ modulo~$n$. Sei~$P$ das Produkt dieser Primzahlen.
\begin{enumerate}
\item Zeige, dass es eine natürliche Zahl~$\ell$ gibt, sodass~$N \defeq
\Phi_n(\ell n P)$ größer als Eins ist.
\item Zeige, dass~$N$ einen Primfaktor~$q$ enthält, welcher ungleich
allen~$p_i$ ist.

{\tiny\emph{Tipp}. Es gilt~$\Phi_n(0) = \pm1$ (wieso?). Was ist daher~$N$
modulo den~$p_i$?\par}
\item Weise nach, dass~$\ell n P$ modulo~$q$ invertierbar ist und in~$\FF_q^\times$
Ordnung~$n$ besitzt.
\item Zeige, dass~$q \equiv 1$ modulo~$n$.
\item[$\heartsuit$ e)] Extrahiere aus diesem Beweis von Dirichlets Satz eine
obere Schranke für die Größe der~$m$-ten Primzahl, welche modulo~$n$ gleich~$1$
ist.
\end{enumerate}
\end{aufgabe}

\begin{aufgabe}{Ein Geheimnis der Zahl 5}
\begin{enumerate}
\item Sei~$x \in \ZZ$. Sei~$p$ eine Primzahl. Zeige: $\leg{x}{p} \equiv
x^{(p-1)/2}$ modulo~$p$.
\item Sei~$p$ eine Primzahl. Sei~$F_p$ die~$p$-te Fibonaccizahl. Zeige: $F_p
\equiv \leg{5}{p}$ modulo~$p$.

{\tiny\emph{Tipp.} Verwende die bekannte Formel~$F_n = (\Phi^n - \Psi^n) /
(\Phi - \Psi)$, wobei~$\Phi = (1 + \sqrt{5})/2$ und~$\Psi = (1 - \sqrt{5})/2$.
Zwei ganze Zahlen teilen einander genau dann, wenn sie es
in~$\O_{\QQ[\sqrt{5}]}$ tun (wieso?).\par}
\end{enumerate}
\end{aufgabe}

\end{document}
