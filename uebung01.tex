\documentclass{uebblatt}

\begin{document}

\maketitle{1}

\begin{aufgabe}{Erste Schritte im Ring der gaußschen Zahlen}
Zerlege folgende Elemente von~$\ZZ[i]$ in irreduzible Faktoren in~$\ZZ[i]$:
\begin{multicols}{2}
\begin{enumerate}
\item $119 - 49i$
\item $153 + 24i$
\end{enumerate}
\end{multicols}
\vspace{-1em}
\end{aufgabe}

\begin{aufgabe}{Ein Beispiel für einen nicht-faktoriellen Ring}
Wir betrachten den Ring~$ \O \defeq \ZZ[\sqrt{-5}] \defeq \{ a + b \sqrt{-5} \,|\, a,b \in \ZZ \}
\subseteq \CC$.
\begin{enumerate}
\item Was sind die Einheiten von~$\O$?
\item Zeige, dass folgende Elemente von~$\O$ alle irreduzibel sind:
\[ 3, \quad 7, \quad 1 + 2\sqrt{-5}, \quad 1 - 2\sqrt{-5}. \]
\item Zeige, dass~$\O$ nicht faktoriell ist, indem du~$21 \in \O$ auf zwei
verschiedene Arten zerlegst.
\end{enumerate}
\end{aufgabe}

\begin{aufgabe}{Ein Beispiel für einen faktoriellen Ring}
Zeige, dass der Ring
$\O \defeq \ZZ[\tfrac{1+\sqrt{-7}}{2}] \defeq
  \{ a + b \tfrac{1+\sqrt{-7}}{2} \,|\, a,b \in \ZZ \} \subseteq \CC$
euklidisch ist.
\end{aufgabe}

\begin{aufgabe}{Geschenkte Ganzzahligkeit rationaler Lösungen}
\begin{enumerate}
\item Zeige, dass eine rationale Zahl genau dann ganzzahlig ist, wenn sie
\emph{ganz über~$\ZZ$} ist, also Nullstelle eines normierten Polynoms mit
ganzzahligen Koeffizienten ist.
\item Zeige damit schnell und mühelos: $\sqrt[n]{2}$ ist für~$n \geq 3$ nicht
rational.
\item Folgere die Behauptung von~b) aus dem Großen Fermatschen Satz. Was ist daran
\emph{besonders} witzig?
\end{enumerate}
\end{aufgabe}

\end{document}
